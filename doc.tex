\documentclass[12pt, a4paper]{article}
\usepackage[polish]{babel}
\usepackage[utf8]{inputenc}
\usepackage[T1]{fontenc}
\frenchspacing
\usepackage{indentfirst}
\usepackage{comment}
\usepackage{hyperref}
\usepackage[hang,flushmargin]{footmisc} 

\hypersetup{
    colorlinks  =   true,                   
    linkcolor   =   black,                  
    citecolor   =   black,                  
    urlcolor    =   blue                    
}
\urlstyle{same}

\title{Mobilne systemy informatyczne \\\normalsize\textbf{Projekt i implementacja systemu mobilnego}}
\author{Krzysztof Radosław Osada}
\date{\today}

\begin{comment}
    TODO:
    * wybór tematu
    * stworzenie projektu
    * wdrożenie
\end{comment}

\begin{document}
    \maketitle

    \tableofcontents

    \section{Wprowadzenie}
    Niniejszy dokument jest omówieniem projektu i implementacji prostego \textbf{systemu mobilnego}. Praca jest zbudowana z czterech głównych sekcji: w~pierwszej z nich zawarty został wstęp do tematu, w drugiej -- informacje na temat projektu, w trzeciej -- dokumentacja ułatwiająca korzystanie z~systemu, natomiast w czwartej -- dane dotyczące implementacji.

    \subsection{Podstawowe definicje}
    W celu odpowiedniego zrozumienia przedstawianego problemu niezbędne wydaje się wypunktowanie kilku istotnych dlań pojęć:
    \begin{itemize}
        \item \textbf{system mobilny} składający się zarówno z elementów stałych, jak i~ruchomych: użytkowników, serwerów oraz stacji bazowych;
        \item \textbf{użytkownicy}, mobilni, najczęściej przemieszczający się, korzystający z urządzeń bezprzewodowych i m.in. pod tym względem różnorodni (mający smartfony, palmtopy czy nawet radiowozy policyjne);
        \item \textbf{serwer}, komputer stacjonarny świadczący usługi na rzecz użytkowników znajdujących się w sieci;
        \item \textbf{stacja bazowa} zapewiająca utrzymanie łączności pomiędzy użytkownikiem a serwerem.
    \end{itemize}

    Wśród cech charakterystycznych systemów mobilnych wymienić można: brak wspólnej pamięci i globalnego zegara, komunikację ograniczoną do wymiany wiadomości czy asynchronizm wykonywanych operacji.

    Połączenia w systemach mobilnych mogą być przewodowe bądź bezprzewodowe -- do tych ostatnich zalicza się: podczerwone, radiowe, ultradźwiękowe, mikrofalowe i laserowe.

    Cennym źródłem informacji na temat systemów mobilnych są materiały dydaktyczne Wydziału Matematyki, Informatyki i Mechaniki Uniwersytetu Warszawskiego\footnote{\ M. Sobczak. Systemy mobilne - Studia Informatyczne. \url{http://wazniak.mimuw.edu.pl/images/1/11/Systemy_mobilne_wyklad_2.pdf}. Dostępny: 24.03.2018.}.

    \subsection{Opis świata rzeczywistego}
    Tutaj przedstawię funkcjonujące już rozwiązania.

    \subsection{Tematyka pracy}
    Tutaj omówię tematykę pracy, wynikającą z powyższego opisu świata rzeczywistego.

    \section{Projekt}
    
    \subsection{Założenia}
    \subsection{Schematy i diagramy}
    \subsection{Interfejs użytkownika}

    \section{Dokumentacja}
    \subsection{Lista funkcjonalności}
    \subsection{Instrukcja użytkownika}

    \section{Implementacja}

    First document. Lorem ipsum dolor sit amet, consecteur adipiscing elit.
    Krzysztof Radosław Osada, syn Renaty Moniki z domu Szecówka i Macieja.
    ą ć ę ł ń ó ś ż ź
\end{document}